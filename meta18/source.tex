\documentclass[sigplan, authordraft]{acmart}

% Copyright
%\setcopyright{none}
%\setcopyright{acmcopyright}
\setcopyright{acmlicensed}
%\setcopyright{rightsretained}
%\setcopyright{usgov}
%\setcopyright{usgovmixed}
%\setcopyright{cagov}
%\setcopyright{cagovmixed}

% DOI
\acmDOI{}

% ISBN
\acmISBN{}

%Conference
\acmConference[META'18]{the 3rd ACM SIGPLAN International Workshop on Meta-Programming Techniques and Reflection}{November 4-9, 2018}{Boston, MA, USA}
\acmYear{2018}
\copyrightyear{2018}
\acmPrice{15.00}
\acmSubmissionID{}

\usepackage{enumerate, enumitem}
\usepackage{fancyvrb}
\fvset{commandchars=\\\{\}}
\usepackage{xcolor}
\usepackage{setspace}
\usepackage{todonotes}
\usepackage{tikz-cd}
\usepackage{adjustbox}
% Idris
% in REPL run `:pp latex 80 <expr>` to get the colored verbatim

\usepackage{amssymb}
\newcommand\fakeslant[1]{%
  \special{pdf: literal 1 0 0.167 1 0 0 cm}#1\special{pdf: literal 1 0 -0.167 1 0 0 cm}}
\definecolor{IdrisRed}{HTML}{BC0045}
\definecolor{IdrisBlue}{HTML}{0000d6}
\definecolor{IdrisGreen}{HTML}{008b00}
\definecolor{IdrisLilac}{HTML}{AC22BF}
\definecolor{IdrisYellow}{HTML}{E16800}
\definecolor{IdrisLightBlue}{HTML}{A4CDFF}
\newcommand{\IdrisData}[1]{\textcolor{IdrisRed}{#1}}
\newcommand{\IdrisType}[1]{\textcolor{IdrisBlue}{#1}}
\newcommand{\IdrisBound}[1]{\textcolor{IdrisLilac}{\fakeslant{#1}}}
\newcommand{\IdrisFunction}[1]{\textcolor{IdrisGreen}{#1}}
\newcommand{\IdrisMetavar}[1]{\textcolor{IdrisYellow}{#1}}
\newcommand{\IdrisKeyword}[1]{{\textbf{#1}}}
\newcommand{\IdrisImplicit}[1]{{\itshape \IdrisBound{#1}}}
\newcommand{\ty}[1]{\IdrisType{\texttt{#1}}}
\newcommand{\kw}[1]{\IdrisKeyword{\texttt{#1}}}
\newcommand{\fn}[1]{\IdrisFunction{\texttt{#1}}}
\newcommand{\dt}[1]{\IdrisData{\texttt{#1}}}
\newcommand{\bn}[1]{\IdrisBound{\texttt{#1}}}
\newcommand{\cm}[1]{\textcolor{darkgray}{\texttt{#1}}}
\newcommand{\hole}[1]{\IdrisMetavar{\texttt{?}\IdrisMetavar{\texttt{#1}}}}
\newcommand{\Elab}{\ty{Elab}}
\newcommand{\IO}{\ty{IO}}
\newcommand{\String}{\ty{String}}
\newcommand{\TT}{\ty{TT}}
\newcommand{\Raw}{\ty{Raw}}
\newcommand{\Editorable}{\ty{Editorable}}
\newcommand{\TyDecl}{\ty{TyDecl}}
\newcommand{\FunDefn}{\ty{FunDefn}}
\newcommand{\FunClause}{\ty{FunClause}}
\newcommand{\Edit}{\ty{Edit}}
\newcommand{\sexp}{\mbox{S-expression}} %don't hyphenate
\newcommand{\mt}[1]{\mbox{\texttt{#1}}}
\newcommand{\Prim}[1]{\dt{Prim\_\_#1}}
\newcommand{\qt}[1]{\kw{\`{}(}{#1}\kw{)}} %quotation `(...)
\newcommand{\antiqt}{\kw{\textasciitilde}} %antiquotation ~

% Replicating selecting regions in the editor:
% \newcommand{\select}[1]{\fcolorbox{black}{IdrisLightBlue}{#1}}
% A version of select without any space around the box.
% Hard to see in black & white so I chose the one with the space
\newcommand{\select}[1]{{\setlength{\fboxsep}{0.05cm}\fcolorbox{black}{IdrisLightBlue}{#1}}}

\newcommand{\Red}[1]{{\color{red} #1}}
\newcommand{\TODO}[1]{{\color{red}{[TODO: #1]}}}
\newcommand{\FYI}[1]{{\color{green}{[FYI: #1]}}}
\newcommand{\forceindent}{\hspace{\parindent}}

\usepackage[normalem]{ulem}
\makeatletter
\newcommand\colorwave[1][blue]{\bgroup \markoverwith{\lower3.5\p@\hbox{\sixly \textcolor{#1}{\char58}}}\ULon}
\font\sixly=lasy6 % does not re-load if already loaded, so no memory problem.
\makeatother
\newcommand{\IdrisError}[1]{\colorwave[red]{#1}}

\sloppy

%% Get rid of sloppy line breaks in \citet citations
\bibpunct{\nolinebreak[4][}{]}{,}{n}{}{,}
\makeatletter
\renewcommand*{\NAT@spacechar}{~}
\makeatother


\setenumerate{label=(\arabic*), parsep=0pt}

\begin{document}
\title[A Metaprogrammable Editor Mode]{A Metaprogrammable Editor Mode}

\author{Joomy Korkut}
\orcid{0000-0001-6784-7108}
\affiliation{
  \institution{Princeton University}
  % \city{Princeton}
  % \state{New Jersey}
  % \country{USA}
}
\email{ckorkut@princeton.edu}

\renewcommand{\shortauthors}{Joomy Korkut}

\begin{abstract}
  Type-directed \emph{editor action}s in languages like Idris and Agda allows programmers to
  ask their editors questions about incomplete programs, and get
  meaningful answers based on types of the missing parts.
  This turns programming into a conversation between the programmer and the
  compiler, where the editor acts as an intermediary.
  Until recently, these languages supported a limited set of questions built-in
  to the editor mode and the compiler.  Recent work on extensible editor
  actions~\cite{extensible} in Idris leveraged metaprogramming
  techniques to specify how the compiler should answer \emph{custom} questions,
  but that required writing code in the editor language (Emacs Lisp, VimL etc.) to
  glue the compiler and the editor interface. Our work allows replacing this
  glue code with code written in Idris itself: We defined a domain-specific
  language in Idris, which provides a monadic interface for editor commands and
  lets users compose these actions in meaningful ways to define more complex
  editor commands, enough to glue the compiler and the editor interface.
\end{abstract}

%
% The code below should be generated by the tool at
% http://dl.acm.org/ccs.cfm
% Please copy and insert the code instead of the example below.
%
 \begin{CCSXML}
<ccs2012>
<concept>
<concept_id>10011007.10011006.10011008.10011009.10011012</concept_id>
<concept_desc>Software and its engineering~Functional languages</concept_desc>
<concept_significance>300</concept_significance>
</concept>
<concept>
<concept_id>10011007.10011006.10011008.10011024</concept_id>
<concept_desc>Software and its engineering~Language features</concept_desc>
<concept_significance>300</concept_significance>
</concept>
</ccs2012>
\end{CCSXML}
\ccsdesc[300]{Software and its engineering~Functional languages}
\ccsdesc[300]{Software and its engineering~Language features}

\keywords{Metaprogramming, dependent types, editors}

\maketitle

\emph{This paper uses colors in the example code.}

\section{Introduction}

Most programming languages with rich type systems come with support for popular
editors like Emacs and Vim. The infrastructure for this support consists of
two different parts~\cite{editTimeTactics}:

\begin{enumerate}
  \item a plugin to the editor, often written in the script language of the
    editor, such as Emacs Lisp or VimL. This is the \emph{front-end} of the
    editor mode.
  \item a separate program that does the heavy lifting of the editing features
    that work with the language itself.  \mt{ghc-mod} in Haskell,
    \mt{agda-mode} in Agda, and the Idris compiler itself would be perfect
    examples for this. This is the \emph{back-end} of the editor mode.
\end{enumerate}

Once the front-end plugin is installed in the editor, it provides features like
syntax highlighting and automatic indentation. Yet for type-directed
interactive editing features, the front-end has the communicate with the
back-end, which can take advantage of the compiler API to give meaningful
results based on the types in the program.

The type-directed interactive editing features available to programmers are
general-purpose editor actions given by the language developers. These actions
include looking up the type of a hole, splitting a pattern into multiple
cases, and searching how a hole can be filled.

While all of these actions make use of the types, the set of available editor
action is fixed; there is no easy way for the user to add a new editor action.
\TODO{the whole spiel about how my work makes it possible to add a new editor
action in Idris itself}

\begin{figure}[H]
\begin{Verbatim}
\fn{upperChar} : \ty{Edit ()}
\fn{upperChar} =
  \kw{do} \fn{setMark}
     \bn{s} <- \fn{getSelection}
     \fn{deactivateMark}
     \fn{insert} (\fn{toUpper} \bn{s})
\end{Verbatim}
\caption{A simple editor function to capitalize the character under cursor}
\label{fig:upperChar}
\end{figure}

When the definition of \fn{upperChar} is desugared and unfolded, i

\begin{figure}[H]
\begin{Verbatim}
\fn{upperChar} : \ty{Edit ()}
\fn{upperChar} =
  \Prim{Bind}
    \Prim{SetMark}
    (\textbackslash{}_ => \Prim{Bind}
             \Prim{GetSelection}
             (\textbackslash{}\bn{s} => \Prim{Bind}
                      \Prim{DeactivateMark}
                      (\textbackslash{}_ => \Prim{Insert} \bn{s})))
\end{Verbatim}
\caption{A simple editor function to capitalize the character under cursor}
\label{fig:upperChar}
\end{figure}

When the compiler receives a command from the editor to run \fn{upperChar}, it will
\begin{enumerate}
  \item look up the type of \fn{upperChar}
  \item ensure that its type is \ty{Edit ()}
  \item look up the definition of \fn{upperChar}
  \item normalize the definition
  \item inspect the constructor at the head
\end{enumerate}


Inspection of the head constructor works in the compiler as such:
\begin{enumerate}
  \item if the head is \Prim{SetMark}:
    \begin{enumerate}[label=(\alph*)]
      \item send a message to the editor that tells it to set mark, i.e.\ start selection
      \item wait for a success message from the editor
      \item return \dt{()} in the \ty{Edit} monad
    \end{enumerate}
  \item if the head is \Prim{DeactivateMark}:
    \begin{enumerate}[label=(\alph*)]
      \item send a message to the editor that tells it to deactivate mark, i.e.\ finish selection
      \item wait for a success message from the editor
      \item return \dt{()} in the \ty{Edit} monad
    \end{enumerate}
  \item if the head is \Prim{GetSelection}:
    \begin{enumerate}[label=(\alph*)]
      \item send a message to the editor that tells it to deactivate mark, i.e.\ finish selection
      \item wait for a success message from the editor, that contains a string
        value of the selection in the editor
      \item return \dt{()} in the \ty{Edit} monad
    \end{enumerate}
  \item if the head is \Prim{Bind}:
    \begin{enumerate}[label=(\alph*)]
      \item send a message for the first action
      \item wait for a success message from the editor,
        value of the selection in the editor
      \item return \dt{()} in the \ty{Edit} monad
    \end{enumerate}
\end{enumerate}


\TODO{the whole spiel about how the compiler is a server }

% \input{introduction}
% \input{design}
% \input{implementation}
% \input{applications}
% \input{relatedwork}
% \input{conclusion}

\section*{Acknowledgments}

\TODO{We would like to thank ...}

\bibliographystyle{ACM-Reference-Format}
\bibliography{paper}

\end{document}
